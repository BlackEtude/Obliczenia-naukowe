\documentclass{classrep}
\usepackage[utf8]{inputenc}

\studycycle{Informatyka, studia dzienne, inż I st.}
\coursesemester{V}
\coursename{Obliczenia naukowe}
\courseyear{2017/2018}
\courseteacher{dr hab. Paweł Zieliński}
\coursegroup{czwartek TN, 11:15}

\author{
  \studentinfo{Agata Jasionowska}{229726}
}

\title{Laboratorium \ppauza Lista 2}
\begin{document}

\maketitle

\section{Zadanie 1}
	\subsection{Opis problemu}
		Zadanie polegało na ponownym rozwiązaniu zadania 5 z listy 1, jednak na nieznacznie zmienionych danych (usunięto ostatnie cyfry w $x_4$ oraz $x_5$. 
	\subsection{Opis rozwiązania}
		W celu obliczenia iloczynów skalarnych użyto kodu zadania 5 listy 1 oraz zmodyfikowanych zgodnie z treścią zadania danych.
	\subsection{Wyniki}
		Poniższa tabela prezentuje uzyskane wyniki dla czterech algorytmów obliczających iloczyn skalarny:
		\begin{table}[!h]
        	\centering
        	\footnotesize
			\sisetup{
				table-number-alignment = right,
				table-figures-integer  = 10,
				table-figures-decimal = 16,
				table-figures-exponent=2
			}
			\begin{tabular}{l|S|S} \toprule
				{podpunkt} & {\texttt{Lista1}} & {\texttt{Lista2}} \\ \midrule
				&\multicolumn{2}{c}{\texttt{Float32}} \\ \midrule
				$1$ & -0.4999443 & -0.4999443 \\ 
	 			$2$ & -0.4543457 & -0.4543457 \\
	 			$3$ & -0.5 & -0.5 \\
	 			$4$ & -0.5 & -0.5 \\
	 			\midrule
	 			&\multicolumn{2}{c}{\texttt{Float64}} \\ \midrule
	 			$1$ & -0.4999443 & -0.004296342739891585 \\ 
	 			$2$ & -0.4543457 & -0.004296342998713953 \\
	 			$3$ & -0.5 & -0.004296342842280865 \\
	 			$4$ & -0.5 & -0.004296342842280865 \\ \bottomrule
	 		\end{tabular}
	 		\caption{Obliczanie iloczynu skalarnego wektorów.}
			\label{table:1}
		\end{table}	
	\subsection{Wnioski}
		Uzyskane wyniki pokazują, że poczynione zmiany nie wpłynęły na rezultaty w arytmetyce \texttt{Float32}. Dzieje się tak z uwagi na stosunkowo niską precyzje obliczeń. Przeprowadzenie 
\section{Zadanie 2}
	\subsection{Opis problemu}
		W co najmniej dwóch wybranych programach do wizualizacji narysować wykres funkcji $f(x)=e^{x}ln(1+e^{-x})$ oraz policzyć granicę $\lim_{x \to \infty} f(x)$.
		%\[ \lim_{x \to \infty} f(x) \]
		
	\subsection{Opis rozwiązania}
	\subsection{Wyniki}			
	\subsubsection{Wnioski}
\section{Zadanie 3}
	\subsection{Opis problemu}
		Rozwiązanie układu równań liniowych $Ax = b$ dla danej macierzy współczynników $A 
		\in {\rm I\!R^{n \times n}}$ i wektora prawych stron $b \in {\rm I\!R^n}$ za 
		pomocą algorytmów: eliminacji Gaussa ($x=A/b$)% oraz $x = A^{-1}b$ %(x=inv(A)*b)$
	\subsection{Opis rozwiązania}
	\subsection{Wyniki}
		
	\subsection{Wnioski}
\section{Zadanie 4}
	\subsection{Opis problemu}
	\subsection{Opis rozwiązania}
	\subsection{Wyniki}
	\subsection{Wnioski}
\section{Zadanie 5}
	\subsection{Opis problemu}
	\subsection{Opis rozwiązania}
	\subsection{Wyniki}
	\subsection{Wnioski}
\section{Zadanie 6}
	\subsection{Opis problemu}
	\subsection{Opis rozwiązania}
	\subsection{Wyniki}
	\subsection{Wnioski}
\end{document}
